\documentclass[11pt]{article}

\usepackage{amsmath,amsfonts,amssymb,fullpage}
\usepackage{booktabs}
\usepackage{enumitem}

\newcommand{\RR}{\mathbb{R}}

\begin{document}
\title{Linear Algebra Requirements for GLM Time Stepping}
\date{\today}
\author{Greg von Winckel}
\maketitle

\section{Classification Dimensions}

The computational requirements for a single time step depend on three orthogonal factors:

\begin{enumerate}
\item \textbf{Stage matrix structure} (from $A$): How are internal stages coupled?
\item \textbf{Propagation matrix structure} (from $V$): How are external stages coupled?
\item \textbf{Problem linearity}: Is $f(y,u,t)$ linear, bilinear, or fully nonlinear in $y$?
\end{enumerate}

\section{Stage Matrix Structure}

The stage equations are:
\[
Z_i = \sum_{j=1}^r u_{ij} y_j^{[n-1]} + h\sum_{j=1}^s a_{ij} f(Z_j, u_j, t+c_jh)
\]

The structure of $A = (a_{ij})$ determines the solve type:

\begin{center}
\begin{tabular}{llll}
\toprule
Structure & Definition & Solve type & Cost \\
\midrule
\textsc{Explicit} & $A$ strictly lower triangular & Forward substitution & $O(s \cdot n \cdot \text{eval})$ \\
\textsc{DIRK} & $A$ lower triangular, $a_{ii} \neq 0$ & $s$ sequential $n\times n$ solves & $O(s \cdot n^3)$ \\
\textsc{SDIRK} & DIRK with $a_{ii} = \gamma$ constant & $s$ solves, 1 factorization & $O(n^3 + s \cdot n^2)$ \\
\textsc{Implicit} & $A$ dense & Coupled $ns \times ns$ solve & $O((ns)^3)$ \\
\bottomrule
\end{tabular}
\end{center}

\paragraph{Detection.} Given tableau $A$:
\begin{verbatim}
def classify_stage_structure(A):
    if is_strictly_lower_triangular(A):
        return EXPLICIT
    if is_lower_triangular(A):
        diag = diagonal(A)
        if all_equal(diag) and diag[0] != 0:
            return SDIRK
        elif all_nonzero(diag):
            return DIRK
        else:
            return EXPLICIT  # zeros on diagonal = explicit stages
    return IMPLICIT
\end{verbatim}

\section{Propagation Matrix Structure}

The external stage update is:
\[
y_i^{[n]} = \sum_{j=1}^r v_{ij} y_j^{[n-1]} + h\sum_{j=1}^s b_{ij} f(Z_j, u_j, t+c_jh)
\]

The structure of $V = (v_{ij})$ determines the propagation type:

\begin{center}
\begin{tabular}{llll}
\toprule
Structure & Definition & Operation & Cost \\
\midrule
\textsc{Identity} & $V = I_r$ & Direct copy + update & $O(r \cdot n)$ \\
\textsc{Shift} & $V$ is a shift matrix & Cyclic buffer update & $O(r \cdot n)$ \\
\textsc{Triangular} & $V$ lower triangular & Forward substitution & $O(r^2 \cdot n)$ \\
\textsc{Dense} & $V$ general & $nr \times nr$ matvec or solve & $O((nr)^2)$ \\
\bottomrule
\end{tabular}
\end{center}

\paragraph{Common cases:}
\begin{itemize}[nosep]
\item Runge--Kutta: $r=1$, $V = (1)$ — trivial
\item BDF/Adams: $V$ is a shift matrix (moves history back one slot)
\item Nordsieck: $V$ encodes Pascal triangle structure
\end{itemize}

\section{Problem Linearity}

\begin{center}
\begin{tabular}{lll}
\toprule
Type & Form of $f$ & Jacobian $F = \partial f/\partial y$ \\
\midrule
\textsc{Linear} & $f(y,u,t) = F(t)y + g(u,t)$ & Constant in $y$, independent of $u$ \\
\textsc{Bilinear} & $f(y,u,t) = (H + uV)y$ & $F = H + uV$, depends on $u$ only \\
\textsc{Quasilinear} & $f(y,u,t) = F(u,t)y + g(u,t)$ & Depends on $u$, not $y$ \\
\textsc{Semilinear} & $f(y,u,t) = Fy + g(y,u,t)$ & Constant (stiff part) + nonlinear \\
\textsc{Nonlinear} & General $f(y,u,t)$ & $F(y,u,t)$ depends on everything \\
\bottomrule
\end{tabular}
\end{center}

\section{Combined Requirements}

For implicit stages with a nonlinear $f$, we need Newton iteration. The table below shows what each combination requires:

\begin{center}
\begin{tabular}{l|ccc}
\toprule
& \textsc{Linear/Bilinear} & \textsc{Quasilinear} & \textsc{Nonlinear} \\
\midrule
\textsc{Explicit} & Matvec & Matvec & Eval \\
\textsc{DIRK} & $s$ solves & $s$ solves & $s \times$ Newton($n$) \\
\textsc{SDIRK} & 1 factor + $s$ solves & $s$ solves$^\dagger$ & $s \times$ Newton($n$)$^\ddagger$ \\
\textsc{Implicit} & 1 block solve & 1 block solve & Newton($ns$) \\
\bottomrule
\end{tabular}
\end{center}

$^\dagger$ SDIRK advantage lost if $F$ changes with $u$ between stages.

$^\ddagger$ Can reuse Jacobian factorization across Newton iterations (modified Newton).

\section{Data Structures for Requirements}

\subsection{Tableau Metadata}

\begin{verbatim}
struct TableauStructure {
    // Stage structure
    enum class StageType { Explicit, DIRK, SDIRK, Implicit };
    StageType stage_type;
    
    // For SDIRK: the common diagonal value
    std::optional<double> sdirk_gamma;
    
    // Indices of explicit stages (where a_{ii} = 0)
    std::vector<int> explicit_stages;
    
    // Propagation structure  
    enum class PropType { Identity, Shift, Triangular, Dense };
    PropType prop_type;
    
    // Dimensions
    int s;  // internal stages
    int r;  // external stages
    
    // Derived
    bool is_fully_explicit() const { 
        return stage_type == StageType::Explicit; 
    }
    bool has_reusable_factorization() const {
        return stage_type == StageType::SDIRK;
    }
};
\end{verbatim}

\subsection{Problem Metadata}

\begin{verbatim}
struct ProblemStructure {
    enum class Linearity { Linear, Bilinear, Quasilinear, Semilinear, Nonlinear };
    Linearity linearity;
    
    // For semilinear: split into stiff (linear) and nonstiff parts?
    bool has_split_rhs;
    
    // Jacobian properties
    bool jacobian_constant;           // F independent of y
    bool jacobian_control_dependent;  // F depends on u
    bool jacobian_time_dependent;     // F depends on t
    
    // What derivatives are available?
    bool has_jacobian;        // F = df/dy
    bool has_control_jacobian; // G = df/du  
    bool has_hessian_action;   // F_yy[v], F_yu[v], F_uu[v]
    
    // Or: use AD for everything?
    bool use_ad;
};
\end{verbatim}

\subsection{Solver Requirements}

\begin{verbatim}
struct SolverRequirements {
    // What operations are needed per time step?
    
    // Evaluation
    int f_evals_per_step;         // Number of RHS evaluations
    int jacobian_evals_per_step;  // Number of Jacobian evaluations
    
    // Factorizations
    int factorizations_per_step;  // LU/Cholesky factorizations
    int factorization_size;       // n or ns
    bool can_reuse_factorization; // Same factorization for all stages?
    
    // Solves (after factorization)
    int solves_per_step;
    int solve_size;               // n or ns
    
    // Newton iterations (if applicable)
    bool needs_newton;
    int newton_system_size;       // n or ns
    int max_newton_iters;
    
    // Memory
    int vectors_to_store;         // Working vectors needed
    int matrices_to_store;        // Jacobians/factorizations to cache
    
    // For adjoint/sensitivity (optimization)
    bool needs_trajectory_storage;  // Store all (Z, y) for backward pass
    int trajectory_storage_size;    // nsN + nrN
};
\end{verbatim}

\section{Dispatch Logic}

\begin{verbatim}
auto deduce_requirements(TableauStructure tab, ProblemStructure prob) 
    -> SolverRequirements 
{
    SolverRequirements req;
    
    // RHS evaluations: always s per step
    req.f_evals_per_step = tab.s;
    
    // Explicit methods: no solves
    if (tab.is_fully_explicit()) {
        req.factorizations_per_step = 0;
        req.solves_per_step = 0;
        req.needs_newton = false;
        return req;
    }
    
    // Implicit + nonlinear: need Newton
    if (prob.linearity == Nonlinear && !tab.is_fully_explicit()) {
        req.needs_newton = true;
        req.newton_system_size = (tab.stage_type == Implicit) ? tab.s * n : n;
        req.max_newton_iters = 10;  // typical default
        
        // Jacobian evals: once per Newton iteration (or reuse = modified Newton)
        req.jacobian_evals_per_step = tab.s;  // at least once per stage
    }
    
    // Linear/bilinear: direct solve, no Newton
    if (prob.linearity <= Quasilinear) {
        req.needs_newton = false;
        
        switch (tab.stage_type) {
            case SDIRK:
                req.factorizations_per_step = prob.jacobian_constant ? 1 : tab.s;
                req.can_reuse_factorization = prob.jacobian_constant;
                req.solves_per_step = tab.s;
                req.factorization_size = n;
                break;
            case DIRK:
                req.factorizations_per_step = tab.s;  // different diagonal each stage
                req.can_reuse_factorization = false;
                req.solves_per_step = tab.s;
                req.factorization_size = n;
                break;
            case Implicit:
                req.factorizations_per_step = 1;
                req.solves_per_step = 1;
                req.factorization_size = tab.s * n;
                break;
        }
    }
    
    return req;
}
\end{verbatim}

\section{Solver Concept Hierarchy}

\begin{verbatim}
// Base operations every problem must provide
template<typename P>
concept Problem = requires(P p, Vector y, Scalar u, Scalar t) {
    { p.f(y, u, t) } -> VectorLike;
    { p.dimension() } -> std::convertible_to<int>;
};

// For implicit methods
template<typename P>
concept DifferentiableProblem = Problem<P> && requires(P p, Vector y, ...) {
    { p.jacobian(y, u, t) } -> MatrixLike;        // F = df/dy
    { p.apply_jacobian(y, u, t, v) } -> VectorLike;  // F*v (matrix-free)
};

// For optimization (first-order)
template<typename P>
concept OptimizableProblem = DifferentiableProblem<P> && requires(P p, ...) {
    { p.control_jacobian(y, u, t) } -> MatrixLike;  // G = df/du
};

// For optimization (second-order)
template<typename P>
concept TwiceDifferentiableProblem = OptimizableProblem<P> && requires(P p, ...) {
    { p.hessian_yy_action(y, u, t, v) } -> MatrixLike;  // F_yy[v]
    { p.hessian_yu_action(y, u, t, v) } -> MatrixLike;  // F_yu[v]
    { p.hessian_uu_action(y, u, t, v) } -> MatrixLike;  // F_uu[v]
};

// Method provides tableau and structure
template<typename M>
concept GLMethod = requires(M m) {
    { m.A() } -> MatrixLike;
    { m.B() } -> MatrixLike;
    { m.U() } -> MatrixLike;
    { m.V() } -> MatrixLike;
    { m.c() } -> VectorLike;
    { m.structure() } -> std::same_as<TableauStructure>;
};
\end{verbatim}

\section{Example: Solver Selection}

\begin{verbatim}
template<GLMethod Method, Problem Prob>
auto make_stepper(Method&& method, Prob&& problem) {
    auto tab = method.structure();
    auto prob = deduce_problem_structure(problem);
    auto req = deduce_requirements(tab, prob);
    
    if (tab.is_fully_explicit()) {
        return ExplicitStepper(method, problem);
    }
    
    if (req.needs_newton) {
        if (tab.stage_type == TableauStructure::Implicit) {
            return FullyImplicitNewtonStepper(method, problem);
        } else {
            return DIRKNewtonStepper(method, problem);
        }
    }
    
    // Linear implicit
    if (tab.stage_type == TableauStructure::SDIRK && req.can_reuse_factorization) {
        return SDIRKLinearStepper(method, problem);  // factor once
    }
    
    return DIRKLinearStepper(method, problem);  // factor per stage
}
\end{verbatim}

\section{IMEX Extension}

For IMEX (additive) methods with $f = f_E + f_I$:

\begin{center}
\begin{tabular}{lll}
\toprule
Component & Tableau & Treatment \\
\midrule
Explicit ($f_E$) & $A^E$ strictly lower triangular & Evaluation only \\
Implicit ($f_I$) & $A^I$ with structure above & Solve as classified \\
\bottomrule
\end{tabular}
\end{center}

The solve structure is determined by $(A^I, \text{linearity of } f_I)$. The explicit part just adds to the RHS before solving.

Requirements struct gains:
\begin{verbatim}
struct IMEXRequirements {
    SolverRequirements explicit_part;  // Always just evals
    SolverRequirements implicit_part;  // Solves as needed
    
    int total_f_evals() const {
        return explicit_part.f_evals_per_step + implicit_part.f_evals_per_step;
    }
};
\end{verbatim}

\section{Summary: The Decision Tree}

\begin{verbatim}
Given: Tableau (A, U, B, V), Problem f(y, u, t)

1. Classify A:
   - Strictly lower triangular? -> EXPLICIT
   - Lower triangular, constant nonzero diagonal? -> SDIRK  
   - Lower triangular, varying diagonal? -> DIRK
   - Dense? -> IMPLICIT

2. Classify f:
   - f linear in y? -> LINEAR
   - f = (H + uV)y? -> BILINEAR
   - Jacobian independent of y? -> QUASILINEAR
   - General? -> NONLINEAR

3. Combine:
   - EXPLICIT + anything: just evaluate f
   - SDIRK + LINEAR: 1 factorization, s solves
   - SDIRK + BILINEAR: s factorizations (u changes), s solves
   - DIRK + LINEAR/BILINEAR: s factorizations, s solves
   - IMPLICIT + LINEAR/BILINEAR: 1 block factorization, 1 block solve
   - any implicit + NONLINEAR: Newton iteration
     - DIRK: s independent Newton solves (size n)
     - IMPLICIT: 1 coupled Newton solve (size ns)

4. For optimization, additionally need:
   - G = df/du at each stage
   - Trajectory storage for backward pass
   - F_yy, F_yu, F_uu contractions for second-order
\end{verbatim}

\section{Linear Algebra Primitives Interface}

The solver needs a linear algebra backend that provides these operations. The key insight: different problem structures need different subsets.

\subsection{Primitive Operations}

\begin{verbatim}
template<typename Scalar, typename Index>
concept LinearAlgebraBackend = requires {
    // Vector operations (always needed)
    Vector axpy(Scalar a, Vector x, Vector y);    // y <- ax + y
    Scalar dot(Vector x, Vector y);
    Scalar norm(Vector x);
    
    // Matrix-vector (for explicit stages, RHS evaluation)
    Vector matvec(Matrix A, Vector x);            // y <- Ax
    Vector matvec_add(Matrix A, Vector x, Vector y); // y <- Ax + y
    
    // For implicit stages: factorization + solve
    Factorization factor(Matrix A);               // LU, Cholesky, etc.
    Vector solve(Factorization F, Vector b);      // x <- A^{-1}b
    
    // For transpose solves (adjoints)
    Vector solve_transpose(Factorization F, Vector b);  // x <- A^{-T}b
    
    // Optional: matrix-free variants
    Vector apply_operator(Operator A, Vector x);
    Vector apply_operator_transpose(Operator A, Vector x);
};
\end{verbatim}

\subsection{What Each Solver Type Needs}

\begin{center}
\begin{tabular}{lccccc}
\toprule
Solver & \texttt{matvec} & \texttt{factor} & \texttt{solve} & \texttt{solve\_transpose} & Block \\
\midrule
Explicit & \checkmark & & & & \\
SDIRK (linear) & \checkmark & \checkmark & \checkmark & \checkmark$^*$ & \\
DIRK (linear) & \checkmark & \checkmark & \checkmark & \checkmark$^*$ & \\
Implicit (linear) & \checkmark & \checkmark & \checkmark & \checkmark$^*$ & \checkmark \\
Newton (any) & \checkmark & \checkmark & \checkmark & \checkmark$^*$ & maybe \\
\bottomrule
\end{tabular}
\end{center}

$^*$ Transpose solve needed for adjoint equations (optimization).

\subsection{Matrix-Free Option}

For large-scale problems, forming and factoring the stage matrix $I - h\gamma F$ may be infeasible. The matrix-free alternative:

\begin{verbatim}
// Instead of:
Matrix J = problem.jacobian(y, u, t);
Matrix A = I - h * gamma * J;
auto LU = factor(A);
Vector dz = solve(LU, residual);

// Use:
auto A_op = [&](Vector v) { 
    return v - h * gamma * problem.apply_jacobian(y, u, t, v);
};
Vector dz = krylov_solve(A_op, residual, preconditioner);
\end{verbatim}

This requires:
\begin{itemize}[nosep]
\item \texttt{apply\_jacobian(y, u, t, v)}: Jacobian-vector product $Fv$
\item \texttt{krylov\_solve}: GMRES, BiCGSTAB, etc.
\item A preconditioner (often approximate factorization of $I - h\gamma F$)
\end{itemize}

For the adjoint: \texttt{apply\_jacobian\_transpose(y, u, t, v)} for $F^\top v$.

\subsection{Concept Hierarchy for Backends}

\begin{verbatim}
// Minimal: explicit methods only
template<typename B>
concept BasicLA = requires(B b, Vector x, Vector y, Scalar a) {
    { b.axpy(a, x, y) } -> VectorLike;
    { b.matvec(A, x) } -> VectorLike;
};

// For implicit linear problems
template<typename B>
concept DirectSolveLA = BasicLA<B> && requires(B b, Matrix A, Vector x) {
    { b.factor(A) } -> Factorization;
    { b.solve(factor, x) } -> VectorLike;
    { b.solve_transpose(factor, x) } -> VectorLike;
};

// For large-scale / matrix-free
template<typename B>
concept IterativeSolveLA = BasicLA<B> && requires(B b, Operator A, Vector x) {
    { b.apply_operator(A, x) } -> VectorLike;
    { b.krylov_solve(A, x, tol) } -> VectorLike;
};

// Full-featured
template<typename B>
concept FullLA = DirectSolveLA<B> && IterativeSolveLA<B>;
\end{verbatim}

\end{document}
